\documentclass[a4paper]{scrreprt}

\usepackage[german]{babel}
\usepackage[utf8]{inputenc}
\usepackage[T1]{fontenc}
\usepackage{ae}
\usepackage{amssymb}
\usepackage{graphicx}
\usepackage{hyperref}

\begin{document}

\title{BS Zusammenfassung}
\author{Benedict Hauck, Fedor Scholz}
\maketitle

\tableofcontents
\vspace{1cm}

\chapter{01aIntro}

\section{What is an Operating System?}

\begin{itemize}
	\item A program that act as an intermediary between a user of a computer and the computer hardware
	\item Goals:
		\begin{itemize}
			\item Execute user programs and make solvung user problems easier
			\item Make the computer system convenient to use
			\item Use the computer hardware in an efficient manner
		\end{itemize}
\end{itemize}

\chapter{01bCProgramming}
\chapter{02OS}

\section{Monolithisches System}
Vorteile:
	\begin{itemize} 
		\item Einfacher Zugriff auf alle Systemdaten 
		\item Kosten von Modulinteraktionen sind niedrig
		\item Erweiterbar über Schnittstellen
		\item Vorhersehbares Verhalten 
	\end{itemize}
Nachteile:
	\begin{itemize}
		\item Kein Schutz zwischen System und Anwendung
		\item Instabil
	\end{itemize}

Beispiele:
	\begin{itemize}
		\item uCLinux, RTOSe, eCos
	\end{itemize}
	
\section{Mehrschichtiger Ansatz}
	\begin{itemize}
		\item Betriebssystem ist in n Schichten aufgeteilt
		\item Jede Schicht kann nur auf die Funktionen und Dienste von niedrigeren Schichten zugreifen 
			\begin{itemize} 
				\item Schicht 0 ist die Hardware
				\item Schicht n ist das Benutzerinterface
			\end{itemize}
		\item Einfachere Migration zwischen Plattformen
		\item Einfachere Evolution der Hardwareplattform
		\item Niedrigere Schichten implementieren Mechanismen
		\item Höhere Schichten implementieren meistens Policies
	\end{itemize}

\begin{center}
\includegraphics[scale=0.15] {schichtenmodell.png} 
\end{center}

Vorteile:
	\begin{itemize}
		\item Jede Schicht kann unabhängig getestet und verfiziert werden
		\item Korrektheit von Schicht n hängt nur von Schicht n-1 ab (einfacheres Debugging, einfachere Wartung)
	\end{itemize}

Nachteile:
	\begin{itemize}
		\item Nur unidirektionaler Schutz
		\item Beiseitige Abhängigkeit von Schichten verhindert strikte Schichtenbildung
	\end{itemize}
Beispiele:
	\begin{itemize}
		\item THE (Dijkstra), Multics(GE), VOCOS(EWSD)
	\end{itemize}

\chapter{03aProcess-management}
\chapter{03bProcess-management-scheduling}
\chapter{04Process-coordination}
\chapter{05aMemoryManagement}
\chapter{05bMemoryManagement}
\chapter{05cMemoryManagement}
\chapter{06aFileSystems}
\chapter{06bFileSystems}
\chapter{07aImplementingFileSystems}
\chapter{07bImplementingFileSystems}
\chapter{08SecondaryStorageStructure}
\chapter{09IoSystems}

\end{document}
